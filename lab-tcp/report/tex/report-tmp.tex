\documentclass[a4paper,12pt]{article}

\input{header.tex}

\title{Отчёт по лабораторной работе \\ <<Механизмы протокола TCP>>}
\author{Здесь Ф.~И.~О}

\begin{document}

\maketitle

\tableofcontents

% Текст отчёта должен быть читаемым!!! Написанное здесь является рыбой.

\section{Установка и разрыв соединения}

Вывод tcpdump при установке соединения от с1 до с6
\begin{Verbatim}
14:36:28.003927 IP 10.10.0.2.36912 > 10.50.0.2.1996: Flags [S], seq 2237042055, win 64240, options [mss 1460,sackOK,TS val 1256467007 ecr 0,nop,wscale 7], length 0
14:36:28.004191 IP 10.50.0.2.1996 > 10.10.0.2.36912: Flags [S.], seq 1393844325, ack 2237042056, win 65160, options [mss 536,sackOK,TS val 2060647298 ecr 1256467007,nop,wscale 7], length 0
14:36:28.004273 IP 10.10.0.2.36912 > 10.50.0.2.1996: Flags [.], ack 1, win 502, options [nop,nop,TS val 1256467007 ecr 2060647298], length 0
\end{Verbatim}

Вывод tcpdump с с3 c включенным MSS clamp при установке соединения от с1 до с6:
\begin{Verbatim}
14:36:28.003979 IP 10.10.0.2.36912 > 10.50.0.2.1996: Flags [S], seq 2237042055, win 64240, options [mss 1460,sackOK,TS val 1256467007 ecr 0,nop,wscale 7], length 0
14:36:28.004168 IP 10.50.0.2.1996 > 10.10.0.2.36912: Flags [S.], seq 1393844325, ack 2237042056, win 65160, options [mss 536,sackOK,TS val 2060647298 ecr 1256467007,nop,wscale 7], length 0
14:36:28.004290 IP 10.10.0.2.36912 > 10.50.0.2.1996: Flags [.], ack 1, win 502, options [nop,nop,TS val 1256467007 ecr 2060647298], length 0
\end{Verbatim}

Вывод tcpdump с c2 при закрытии соединения на с1
\begin{Verbatim}
14:36:28.003979 IP 10.10.0.2.36912 > 10.50.0.2.1996: Flags [S], seq 2237042055, win 64240, options [mss 1460,sackOK,TS val 1256467007 ecr 0,nop,wscale 7], length 0
14:36:28.004168 IP 10.50.0.2.1996 > 10.10.0.2.36912: Flags [S.], seq 1393844325, ack 2237042056, win 65160, options [mss 536,sackOK,TS val 2060647298 ecr 1256467007,nop,wscale 7], length 0
14:36:28.004290 IP 10.10.0.2.36912 > 10.50.0.2.1996: Flags [.], ack 1, win 502, options [nop,nop,TS val 1256467007 ecr 2060647298], length 0
\end{Verbatim}

Вывод tcpdumpс c2 при закрытии соединения на с1:
\begin{Verbatim}
14:43:08.035884 IP 10.10.0.2.36912 > 10.50.0.2.1996: Flags [F.], seq 8, ack 8, win 502, options [nop,nop,TS val 1256867039 ecr 2061037865], length 0
14:43:08.036067 IP 10.50.0.2.1996 > 10.10.0.2.36912: Flags [F.], seq 8, ack 9, win 510, options [nop,nop,TS val 2061047330 ecr 1256867039], length 0
14:43:08.036111 IP 10.10.0.2.36912 > 10.50.0.2.1996: Flags [.], ack 9, win 502, options [nop,nop,TS val 1256867039 ecr 2061047330], length 0
\end{Verbatim}

Вывод tcpdumpс c3 c включенным MSS clamp при закрытии соединения на с1:
\begin{Verbatim}
14:43:08.035905 IP 10.10.0.2.36912 > 10.50.0.2.1996: Flags [F.], seq 8, ack 8, win 502, options [nop,nop,TS val 1256867039 ecr 2061037865], length 0
14:43:08.036050 IP 10.50.0.2.1996 > 10.10.0.2.36912: Flags [F.], seq 8, ack 9, win 510, options [nop,nop,TS val 2061047330 ecr 1256867039], length 0
14:43:08.036120 IP 10.10.0.2.36912 > 10.50.0.2.1996: Flags [.], ack 9, win 502, options [nop,nop,TS val 1256867039 ecr 2061047330], length 0
\end{Verbatim}

\section{Окно получателя}

Где что дампим.  Дампить без -t обязательно!

\begin{Verbatim}
окно получателя до нуля
\end{Verbatim}

\section{Окно отправителя}

Где что дампим.

\begin{Verbatim}
окно отправителя растёт и растёт
\end{Verbatim}

\section{Нейгл и Мишналь}

Где что дампим.  Дампить без -t обязательно!
Мы должны увидеть, что мы посылаем неполный сегмент, как только подтверждены все неполные сегменты (даже если полные не подтверждены).

\begin{Verbatim}
увидеть хотя бы нейгла
\end{Verbatim}

\section{Аггрессивная буферизация}

Где что дампим.  Дампить без -t обязательно!

\begin{Verbatim}
cork
\end{Verbatim}

\section{Отправка без задержки}

Где что дампим.  Дампить без -t обязательно!

\begin{Verbatim}
увидеть, что nodelay не помогает, когда окно отправителя полное
\end{Verbatim}

\section{Быстрый повтор}

Где что дампим.  Дампить без -t обязательно!

\begin{Verbatim}
увидеть быстрый повтор
\end{Verbatim}

\section{Обычный повтор}

Где что дампим. Дампить без -t обязательно!

\begin{Verbatim}
увидеть не-быстрый повтор
\end{Verbatim}

\section{Неудачная попытка соединени с портом}

Тут опт с попыткой соединения с портом, который никто не слушает.

\section{Опыт с PMTU}

Для опыта нужно на c3 и c4 отключить tcpclump.
Надеюсь тут поможет команда iptables -F.
Для сброса кеша MSS на с1 его можно тупо перегрузить.

После этих подготовительных действий мы наверное увидим pmtu в действии при попытке соединится с c1 на c6.

\begin{Verbatim}
Дампить (tcp or icmp) на c2 eth0.
\end{Verbatim}

\section{Соединение с неверным портом}

Что будет, если клиент пытается соединиться с портом, который не слушвет сервер.

\begin{Verbatim}
Дампить
\end{Verbatim}

\end{document}
